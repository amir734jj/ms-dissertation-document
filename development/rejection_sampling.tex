\subsection{Rejection sampling}
Rejection method produces exactly independent samples with the exact probability density specified, but the number of steps needed to do this cannot be exactly predicted in advance. There is considerable freedom in designing a rejection algorithm to sample a given probability density function, $\rho$.  

The Monte Carlo practitioner occasionally must spend some time and (not that pleasant) effort optimizing parameters or otherwise tinkering to get a rejection  method that is reasonably efficient. Rejection methods are often in the innermost loop of a Monte Carlo code, so their efficiency determines the running time of the code as a whole.  

To generate a random variable with density $\rho(x)$, the rejection method uses independent random variables sampled from an auxiliary density, $\rho_0(x)$ and a \textit{acceptance probability}, $p(x)$.

The method has two steps
\begin{description}
    \item[Trial:] generate a "trial" random variable, $X \sim \rho_0$. All trials are independent.
    \item[Rejection:] "accept" the trial with probability $p(X)$. If $X$ is accepted, it is the random variable generated by the algorithm. If $X$ is rejected, go back to the trial step, generate a new (independent) $X$.
\end{description}

Accepting with probability $p$ is done on the computer by comparing $p$ to another (independent) uniform random variable.  If $p$ is larger (an event with probability $p$), accept. This trial and rejection process is repeated until a random variable is accepted. If $\zeta$ is the probability of getting an acceptance on any given trial, then the expected number of trials needed to get an acceptance
is $1/\zeta$. 

We can determine the probability density function for the eventual accepted $X$ using the laws of conditional probability.  It is given by:

\begin{eqnarray*}
\rho(x)dx & = & \mbox{Prob}\left[ \mbox{ accepted $X \in (x,x+dx)$} \right] \\
          & = & \mbox{Prob}\left[ \mbox{ trial $X \in (x,x+dx)$ } 
              | \mbox{ accepted $X$} \right]                          \\
          & = & \frac{ \mbox{Prob}\left[ \mbox{ trial $X \in (x,x+dx)$ 
                                and accepted $X$ } \right] }
                     { \mbox{Prob}\left[ \mbox{ got an acceptance }\right] } \\
         & = & \frac{1}{\zeta}\rho_0(x)dx\cdot p(x) \;\; ,
\end{eqnarray*}

where $\zeta$ is the probability of getting an acceptance on a given trial, as above. Putting this together gives:

\begin{equation}
p(x) = \zeta \frac{\rho(x)}{\rho_0(x)} \;\; .
\end{equation}

In order for $p(x)$ as given in above to be a probability, it must be between $0$ and $1$.  In order for this to be possible (with a fixed $\zeta$), we must have:

\begin{displaymath}
\rho(x) \leq \frac{1}{\zeta} \rho_0(x) \;\; .
\end{displaymath}

This requirement limits the possibilities for $\rho_0$.  For example, one can sample a standard normal by rejection from an exponential (the ratio $e^{-x^2/2}/e^{-x}$ is bounded), but one cannot sample an exponential by rejection from a Gaussian (the ratio $e^{-x}/e^{-x^2/2}$ is not).
