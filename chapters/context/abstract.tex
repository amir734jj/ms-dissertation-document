\newcommand{\abstractTitleSize}{\normalsize}
{
	
\begin{center}
	\singlespacing
	
	{\abstractTitleSize \MakeUppercase{Abstract}}
	\vspace{0.7cm}
				
	{\abstractTitleSize \MakeUppercase{Analysis of BCNS and NewHope}}
				        
% 	\vspace{0.1cm}
				        
	{\abstractTitleSize \MakeUppercase{Key-Exchange Protocols}}
				        
	\vspace{0.7cm}
				
	{\abstractTitleSize by}
				        
	\vspace{0.5cm}
				        
	{\abstractTitleSize Seyedamirhossein Hesamian}
				        
	\vspace{1cm}   
				    
	{\abstractTitleSize The University of Wisconsin-Milwaukee, 2017}\\
				
	{\abstractTitleSize Under the Supervision of Dr. Guangwu Xu}
				
	\vspace{0.5cm}
\end{center}
	

\doublespacing

\indent Lattice-based cryptographic primitives are believed to offer resilience against attacks by quantum computers. Following increasing interest from both companies and government agencies in building quantum computers, a number of works have proposed instantiations of practical post-quantum key-exchange protocols based on hard problems in lattices, mainly based on the Ring Learning With Errors (R-$\mathrm{LWE}$) problem.


In this work we present an analysis of Ring-$\mathrm{LWE}$ based key-exchange mechanisms and compare two implementations of Ring-$\mathrm{LWE}$ based key-exchange protocol: $\mathrm{BCNS}$ and $\mathrm{NewHope}$. This is important as $\mathrm{NewHope}$ protocol implementation outperforms state-of-the art elliptic curve based Diffie-Hellman key-exchange $\mathrm{X25519}$, thus showing that using quantum safe key-exchange is not only a viable option but also a faster one. Specifically, this thesis compares different reconciliation methods, parameter choices, noise sampling algorithms and performance.
	

		
% 	\noindent \keywords{lattice based key-exchange, Ring-$\mathrm{LWE}$, $\mathrm{BCNS}$, $\mathrm{NewHope}$}


% \thispagestyle{empty}

}
\newpage
