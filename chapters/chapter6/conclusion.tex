$\mathrm{LWE}$ and Ring-$\mathrm{LWE}$ promise a quantum safe passive key-exchange and recent implementations not only offer a drop-in replacement for cryptography software libraries but they also outperform the existing key-exchange algorithm as well. Understanding the concept behind Ring-$\mathrm{LWE}$ problem and how it relates to lattices and lattice problems (e.g $\mathrm{SVP}$ problem) is an essential precursor to understand the Ring-$\mathrm{LWE}$ based key-exchange protocols. Reconciliation algorithms plays a key role that enables the key-exchange protocols to extract the same bits with very small probability of failure. 

This thesis provides a brief survey on lattice based cryptography how it relates to $\mathrm{LWE}$ problem. Thereafter, it discusses the reason behind creation of Ring-$\mathrm{LWE}$ which provides a structure to $\mathrm{LWE}$ matrix of module finite field by introducing concept of quotient polynomials modulo finite field. Then discussion of the basics of Ring-$\mathrm{LWE}$ key-exchange and the need for reconciliation. Next, it reviewed parameter choices and compared the specifics of different reconciliation algorithms. Further, it reviews the two real-world and highly optimized implementations of Ring-$\mathrm{LWE}$ key-exchange protocols; in details, noise sampler, ways to generate shared polynomial and polynomial multiplication algorithm. In the end, it discusses the performance analysis and possibility of using Hybrid ($\mathrm{ECC} + \text{Ring-}\mathrm{LWE}$) approach as opposed to only Ring-$\mathrm{LWE}$.


In the future work, Ring-$\mathrm{LWE}$ based key-exchange might benefit from improved reconciliation algorithms that are simpler to implement yet yield higher success probability. Current key-exchange protocol such as elliptic curve Diffie Hellman and RSA have been studied extensively and their shortcoming have been identified and addressed. For example, Wiener's attack (small private key $d$) and Coppersmith's attack (small public exponent $e$) for RSA. However, lattice based algorithm are relatively new and the claim of security against quantum computers and even classical computers needs to be studied further as field of quantum computing grows.

Lastly, more work needs to be done to standardize key-exchange protocols and their efficient implementations in various applications, frameworks and programming languages (e.g. Ring-$\mathrm{LWE}$ key-exchange as a session-key in context of client–server model inside web frameworks). Exploring other candidates for post-quantum cryptography can also be done as future works. For example, McEliece cryptosystem, NTRU encryption, Supersingular isogeny key-exchange and $\mathrm{LWE}$ based key-exchange (without a ring) and attempts to make them more efficient in terms of both space and time complexity. 
