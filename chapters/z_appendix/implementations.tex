\chapter{Basic implementations of all Ring-\texorpdfstring{$\mathrm{LWE}$}{LWE} key-exchange reconciliations}

One of the goals of this thesis was to implement a simple to understand, easy to modify and straightforward implementation of all four reconciliation mechanisms (i.e. Regev, Ding, Peikert and NewHope). However, after struggling with a well known number theory libraries such as $\mathrm{NTL}$ and $\mathrm{FLINT}$ which are indeed great packages written in $\mathrm{C}$ but mainly for ones who have studied and used number theory for a long time and speed is their primary concern, we selected Python language. Python has a syntax similar to pseudo-code but setting up an environment in Python to use all the libraries needed for the reconciliation implementation \textit{and} achieve three criteria above would not be possible (i.e. setting up polynomial library, then configure quotient ring and etc.). Also, Python's popular mathematical libraries such as $\mathrm{numpy}$ do not offer a quotient Polynomial ring modulo finite field without using a work around like calling $\mathrm{NTL}$ library which is written in $\mathrm{C}$. After searching further we came upon SageMath which uses a Python-like syntax, supporting procedural, functional and object-oriented constructs.

W. Stein, creator of SageMath who is a mathematician at the University of Washington, realized when designing Sage that there were many open-source mathematics software packages already written in different languages, namely $\mathrm{C}$, $\mathrm{C}$++, Common Lisp, Fortran and Python. Rather than reinventing the wheel, Sage which is written mostly in Python integrates many specialized mathematics software packages into a common interface, for which a user needs to know only Python. However, Sage contains hundreds of thousands of unique lines of code adding new functions and creating the interface between its components.

SageMath is an open-source project and it truly a powerful tool but it does not receive enough attention as it deserves specially for it's powerful polynomial package. We moved away from $\mathrm{java}$ after trying to implement a univariate polynomial ring using $\mathrm{java}$ (i.e. reinventing the wheel) and seeing how difficult it is to implement a proper polynomial parser to works with edge cases, although it was a rewarding experience, it was time consuming. We should also give Sage credit for providing comprehensive Matrix library which includes $\mathrm{LLL}$ (by calling $\mathrm{NTL}$'s $\mathrm{LLL}$ implementation) and many other implementations of lattice algorithms (e.g. $\mathrm{BKZ}$, $\mathrm{SVP}$, $\mathrm{CVP}$ and more).

Sage uses many open source libraries and links them all together using Python. It is a simple idea but it requires clever implementation which Sage truly delivers. It is fascinating to see a smart type system that can handle the linkage between all those open-source libraries.

\iftoggle{long}{The following are demonstration of Peikert and Alkim et al. reconciliation algorithms using SageMath. Note that these are not efficient implementations and they only serve as a platform to modify and experiment with the reconciliation algorithms. }

In all implementations\iftoggle{long}{ below}, we used Sage's built in function: \texttt{\justifyx DiscreteGaussianDistributionPolynomialSampler} which realizes oracles which returns polynomials in $\mathbb{Z}[x]$ where each coefficient is sampled independently with a probability proportional to $e^{(-(x-c)^2/(2 \sigma ^2))}$. Thereafter, we modified the ring of returned polynomial to quotient polynomial ring over finite field of integers\iftoggle{long}{ using \texttt{Y(f)}}.

For the complete code and all four reconciliation methods please refer to:

\begin{center}
\href{https://github.com/amir734jj/LWE-KEX}{\texttt{https://github.com/amir734jj/LWE-KEX}}
\end{center}


\iftoggle{long}{
The following are implementations of reconciliation algorithms not protocols. For example, with regards to the $\mathrm{NewHope}$ implementation, we used the Gaussian distribution sampler as oppose to Binomial distribution that $\mathrm{NewHope}$ protocol uses. 

\lstinputlisting[language=Python, caption=Configuration of quotient polynomial ring over finite field using SageMath]{code_snippets/LWE-KEX/quotient_ring.sage}


Below is a tester to check if reconciliation algorithm works correctly. This can be used in conjunction with any reconciliation algorithm. 

\lstinputlisting[language=Python, caption=Tester for reconciliations using SageMath]{code_snippets/LWE-KEX/test.sage}


Below is the implementation of Peikert's reconciliation algorithm. Notice the randomized doubling function that multiplies every coefficient of polynomial by two and then adds a secondary noise. $\mathrm{BCNS}$ protocol samples two bits and use it as an index to uniformly sample from: $\{-1, 0, 0, 1\}$.


\lstinputlisting[language=Python, caption=Peikert reconciliation implementation using SageMath]{code_snippets/LWE-KEX/peikert_rec.sage}


Below is the implementation of $\mathrm{NewHope}$ reconciliation algorithm. We did not implement the concept of splitting Voronoi cell into sub-cells as it is part of $\mathrm{NewHope}$ protocol not the reconciliation. Notice the structure is basically the same as Peikert's reconciliation (i.e. $\texttt{dbl}$, $\texttt{generate\_signal}$, $\texttt{reconcile}$). The $\texttt{initialize}$ function will return an integer lattice created using identity matrix of dimension = $\texttt{sub\_dimension}$. It also returns a polyhedron (or voronoi cell) centered at $(1/2, \dots, 1/2)$ inside a lattice with the following basis (when $\texttt{sub\_dimension = 4}$): $\begin{psmallmatrix}
    1   &   0   &   0   &   0\\
    0   &   1   &   0   &   0\\
    0   &   0   &   1   &   0\\
    0.5   &   0.5   &   0.5   &   0.5
\end{psmallmatrix}$

\lstinputlisting[language=Python, caption=NewHope reconciliation implementation using SageMath]{code_snippets/LWE-KEX/newhope_rec.sage}

}


% \lstinputlisting[language=Python, caption=Regev KEX implementation using SageMath]{code_snippets/LWE-KEX/Regev.sage}

% \lstinputlisting[language=Python, caption=DXL implemented using SageMath]{code_snippets/LWE-KEX/Ding.sage}

% \lstinputlisting[language=Python, caption=Peikert's KEM (BCNS) implemented using SageMath]{code_snippets/LWE-KEX/Peikert.sage}

% \lstinputlisting[language=Python, caption=NewHope implemented using SageMath]{code_snippets/LWE-KEX/NewHope.sage}
