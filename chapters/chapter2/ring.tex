\chapter{Quotient Rings and Field Extensions}

In this section we describe a method for producing field extension of a given
field. If $F$ is a field, then a
\index{field extension}%
\emph{field extension} is a field $K$ that contains $F $. For example,
$\mathbb{C}$ is a field extension of $\mathbb{R}$ since $\mathbb{C}$ is a
field containing $\mathbb{R}$. Similarly, $\mathbb{C}$ is a field extension of
$\mathbb{Q}$. For coding theory we need field extensions of $\mathbb{Z}_{2}$.
To produce a field extension of a field $F$ we will use a polynomial $f(x)$
with coefficients in $F$, and we will produce it by mimicing the idea of
producing the integers modulo $n$ by starting with the integers and a fixed
integer $n$. In order to do this we need to know that the arithmetic of
polynomials is sufficiently similar to the arithmetic of integers. In the
first section of this chapter we see that notions relating to divisibility
work just as well for polynomials over a field as for the integers.

\section{Arithmetic of Polynomial Rings}

Let $F$ be a field, and let $F[x]$ be the ring of polynomials in the
indeterminate $x$. High school students study the arithmetic of this ring
without saying so in so many words, at least for the case $F=\mathbb{R}$. In
this section we make a formal study of this arithmetic, seeing that much of
what we did for integers above can be done in the ring $F[x]$. We start with
the most basic definition.

\begin{definition}
Let $f$ and $g$ be polynomials in $F[x]$. Then we say that $f$ divides $g$, or
$g$ is divisible by $f$, if there is a polynomial $h$ with $g=fh$.
\end{definition}

The greatest common divisor of two integers $a$ and $b$ is the largest integer
dividing both $a$ and $b$. This definition needs to be modified a little for
polynomials. While we cannot talk about \textquotedblleft
largest\textquotedblright\ polynomial in the same manner as for integers, we
can talk about the degree of a polynomial. Recall that the
\index{degree of a polynomial}%
\emph{degree} of a nonzero polynomial $f$ is the largest integer $m$ for which
the coefficient of $x^{m}$ is nonzero. If $f(x)=a_{n}x^{n}+\cdots+a_{1}%
x+a_{0}$ with $a_{n}\neq0$, then the degree of $f(x)$ is $n$. We write
$\deg(f)$ for the degree of $f$. The degree function allows us to measure size
of polynomials. However, there is one extra complication. For example, any
polynomial of the form $ax^{2}$ with $a\neq0$ divides $x^{2}$ and $x^{3}$.
Thus, there isn't a unique polynomial of highest degree that divides a pair of
polynomials. To pick one out, we consider
\index{monic polynomial}%
\emph{monic} polynomials, polynomials whose
\index{leading coefficient}%
\emph{leading coefficient} is $1$. For example, $x^{2}$ is the monic
polynomial of degree $2$ that divides both $x^{2}$ and $x^{3}$, while $5x^{2}$
is not monic. As a piece of terminology, we will refer to an element $f\in
F[x]$ as a \emph{polynomial over }$F$.

\begin{definition}
Let $f$ and $g$ be polynomials over $F$, not both zero. Then a greatest common
divisor of $f$ and $g$ is a monic polynomial of largest degree that divides
both $f$ and $g$.
\end{definition}

The problem with the definition above has to do with uniqueness. Could there
be more than one greatest common divisor of a pair of polynomials? The answer
is no, and we will prove this after we prove the analogue of the division algorithm.

The main reason for assuming that the coefficients of our polynomials lie in a
field is to ensure that the division algorithm is valid. Before we prove it,
we need a simple lemma about degrees. For convenience, we set $\deg
(0)=-\infty$. We also make the convention that $-\infty+-\infty=-\infty$ and
$-\infty+n=-\infty$ for any integer $n$. The point of these conventions is to
make the statement in the following lemma and other results as simple as possible.

\begin{lemma}
\label{degree}Let $F$ be a field and let $f$ and $g$ be polynomials over $F$.
Then $\deg(fg)=\deg(f)+\deg(g)$.
\end{lemma}

\begin{proof}
If either $f=0$ or $g=0$, then the equality $\deg(fg)=\deg(f)+\deg(g)$ is true
by our convention above. So, suppose that $f\neq0$ and $g\neq0$. Write
$f=a_{n}x^{n}+\cdots+a_{0}$ and $g=b_{m}x^{m}+\cdots+b_{0}$ with $a_{n}\neq0$
and $b_{m}\neq0$. Therefore, $\deg(f)=n$ and $\deg(g)=m$. The definition of
polynomial multiplication yields
\[
fg=(a_{n}b_{m})x^{n+m}+(a_{n}b_{m-1}+a_{n-1}b_{m})x^{n+m-1}+\cdots+a_{0}%
b_{0}.
\]
Now, since the coefficients come from a field, which has no zero divisors, we
can conclude that $a_{n}b_{m}\neq0$, and so $\deg(fg)=n+m=\deg(f)+\deg(g)$, as desired.
\end{proof}


\begin{proposition}
[Division Algorithm]%
\index{division algorithm for polynomials}%
Let $F$ be a field and let $f$ and $g$ be polynomials over $F$ with $f$
nonzero. Then there are unique polynomials $q$ and $r$ with $g=qf+r$ with
$\deg(r)<\deg(f)$.
\end{proposition}

\begin{proof}
Let
\[
\mathcal{S}=\left\{  t\in F[x]:t=g-qf\text{ for some }q\in F[x]\right\}  .
\]
Then $\mathcal{S}$ is a nonempty set of polynomials, since $g\in\mathcal{S}$.
Thus, by the well ordering property of the integers, there is a polynomial $r$
of least degree in $\mathcal{S}$. By definition, there is a $q\in F[x]$ with
$r=g-qf$, so $g=qf+r$. We need to see that $\deg(r)<\deg(f)$. If, on the other
hand, $\deg(r)\geq\deg(f)$, say $n=\deg(f)$ and $m=\deg(r)$. If $f=a_{n}%
x^{n}+\cdots+a_{0}$ and $r=r_{m}x^{m}+\cdots+r_{0}$ with $a_{n}\neq0$ and
$r_{m}\neq0$, then by thinking about the method of long division of
polynomials, we realize that we may write $r=(r_{m}a_{n}^{-1})x^{m-n}%
f+r^{\prime}$ with $\deg(r^{\prime})<m=\deg(r)$. But then%
\[
g=qf+r=qf+(r_{m}a_{n}^{-1})x^{m-n}f+r^{\prime}=(q+r_{m}a_{n}^{-1}%
x^{m-n})f+r^{\prime},
\]
which shows that $r^{\prime}\in\mathcal{S}$. Since $\deg(r^{\prime})<\deg(r)$,
this would be a contradiction to the choice of $r$. Therefore, $\deg
(r)\geq\deg(f)$ is false, so $\deg(r)<\deg(f)$, as we wanted to prove. This
proves existence of $q$ and $r$. For uniqueness, suppose that $g=qf+r$ and
$g=q^{\prime}f+r^{\prime}$ for some polynomials $q,q^{\prime}$ and
$r,r^{\prime}$ in $F[x]$, and with $\deg(r),\deg(r^{\prime})<\deg(f) $. Then
$qf+r=q^{\prime}f+r^{\prime}$, so $(q-q^{\prime})f=r^{\prime}-r$. Taking
degrees and using the lemma, we have
\[
\deg(q^{\prime}-q)+\deg(f)=\deg(r^{\prime}-r).
\]
Since $\deg(r)<\deg(f)$ and $\deg(r^{\prime})<\deg(f)$, we have $\deg
(r^{\prime}-r)<\deg(f)$. However, if $\deg(q^{\prime}-q)\geq0$, this is a
contradiction to the equation above. The only way for this to hold is for
$\deg(q^{\prime}-q)=\deg(r^{\prime}-r)=-\infty$. Thus, $q^{\prime
}-q=0=r^{\prime}-r$, so $q^{\prime}=q$ and $r^{\prime}=r$, proving uniqueness.
\end{proof}


We now prove the existence of greatest common divisors of polynomials, and
also prove the representation theorem analogous to Proposition
\ref{gcdrepforintegers}

\begin{proposition}
Let $F$ be a field and let $f$ and $g$ be polynomials over $F$, not both zero.
Then $\gcd(f,g)$ exists and is unique. Furthermore, there are polynomials $h$
and $k$ with $\gcd(f,g)=hf+kg$.
\end{proposition}

\begin{proof}
We will prove this by proving the representation result. Let
\[
\mathcal{S}=\left\{  hf+kg:h,k\in F[x]\right\}  .
\]
Then $\mathcal{S}$ contains nonzero polynomials as $f=1\cdot f+0\cdot g$ and
$g=0\cdot f+1\cdot g$ both lie in $\mathcal{S}$. Therefore, there is a nonzero
polynomial $d\in\mathcal{S}$ of smallest degree by the well ordering
principle. Write $d=hf+kg$ for some $h,k\in F[x]$. By dividing by the leading
coefficient of $d$, we may assume that $d$ is monic without changing the
condition $e\in\mathcal{S}$. We claim that $d=\gcd(f,g)$. To show that $e$ is
a common divisor of $f$ and $g$, first consider $f$. By the division
algorithm, we may write $f=qd+r$ for some polynomials $q$ and $r$, and with
$\deg(r)<\deg(d)$. Then
\begin{align*}
r  & =f-qd=f-q(hf+kg)\\
& =(1-qh)f+(-qk)g.
\end{align*}
This shows $r\in\mathcal{S}$. If $r\neq0$, this would be a contradiction to
the choice of $d$ since $\deg(r)<\deg(d)$. Therefore, $r=0$, which shows that
$f=qd$, and so $d$ divides $f$. Similarly, $d$ divides $g$. Thus, $d$ is a
common divisor of $f$ and $g$. If $e$ is any other common divisor of $f$ and
$g$, then $e$ divides any combination of $f$ and $g$; in particular, $e$
divides $hf+kg=d$. This forces $\deg(e)\leq\deg(d)$ by Lemma \ref{degree}.
Thus, $d$ is the monic polynomial of largest degree that divides $f$ and $g$,
so $d$ is a greatest common divisor of $f$ and $g$. This proves everything but
uniqueness. For that, suppose that $d$ and $d^{\prime}$ are both monic common
divisors of $f$ and $g$ of largest degree. By the proof above, we may write
both $d$ and $d^{\prime}$ as combinations of $f$ and $g$. Also, from this, the
argument above shows that $d$ divides $d^{\prime}$ and vice-versa. If
$d^{\prime}=ad$ and $d=bd^{\prime}$, then $d=bd^{\prime}=abd$. Taking degrees
shows that $\deg(ab)=0$, which means that $a$ and $b$ are both constants. But,
since $d$ and $d^{\prime}$ are monic, for $d^{\prime}=ad$ to be monic, $a=1$.
Thus, $d^{\prime}=ad=d$.
\end{proof}


\section{Ideals and Quotient Rings}

We will construct extension fields of a field $F$ by starting with an ideal of
the polynomial ring $F[x]$ and constructing the associated quotient ring. We
must therefore begin by defining ideals.

\begin{definition}%
\index{ideal}%
Let $R$ be a ring. An ideal $I$ is a nonempty subset of $R$ such that (i) if
$a,b\in I$, then $a+b\in I$, and (ii) if $a\in I$ and $r\in R$, then $ar\in I$
and $ra\in I$.
\end{definition}

This definition says that an ideal is a subset of $R$ closed under addition
that satisfies a strengthened form of closure under multiplication. Not only
is the product of two elements of $I$ also in $I$, but that the product of an
element of $I$ and any element of $r$ is an element of $I$.

\begin{example}
Let $R=\mathbb{Z}$. If $n$ is an integer, let $n\mathbb{Z}$ be the set of all
multiples of $\mathbb{Z}$. That is,
\[
n\mathbb{Z}=\left\{  na:a\in\mathbb{Z}\right\}  .
\]
To see that this set is an ideal, first consider addition. if $x,y\in
n\mathbb{Z}$, then there integers $a$ and $b$ with $x=na$ and $y=nb$. Then
$x+y=na+nb=n(a+b)$. Therefore, $x+y\in n\mathbb{Z}$. Second, for
multiplication, let $x=na\in n\mathbb{Z}$ and let $r\in\mathbb{Z}$. Then
$rx=xr=r(na)=n(ra)$, a multiple of $n$. Therefore, $rx\in n\mathbb{Z}$. This
proves that $n\mathbb{Z}$ is an ideal. If $n>0$, notice that
\[
n\mathbb{Z=}\left\{  0,n,2n,\ldots,-n,-2n,\ldots\right\}
\]
is the same as the equivalence class of $0$ under the relation congruence
modulo $n$. This is an important connection that we will revisit.
\end{example}

\begin{example}
Let $R=F[x]$ be the ring of polynomials over a field, and let $f\in F[x]$.
Let
\[
I=\left\{  gf:g\in F[x]\right\}  ,
\]
the set of all multiples of $f$. This set is an ideal of $F[x]$ by the same
calculation as in the previous example. However, we repeat this calculation.
For closure under addition, let $h,k\in I$. Then $h=gf$ and $k=g^{\prime}f$
for some polynomials $g$ and $g^{\prime}$. Then $h+k=gf+g^{\prime
}f=(g+g^{\prime})f$, a multiple of $f$, so $h+k\in I$. For multiplication, let
$h=gf\in I$, and let $a\in F[x]$. Then $ah=ha=agf=(ag)f$, a multiple of $f$,
so $ah\in I$. Thus, $I$ is an ideal of $F[x]$. This ideal is typically denoted
by $(f)$.
\end{example}

\begin{example}
Let $R$ be any commutative ring, and let $a\in R$. Let
\[
aR=\left\{  ar:r\in R\right\}  .
\]
We can consider $aR$ to be the set of multiples of $a$. We show that $aR$ is
an ideal of $R$. First, let $x,y\in aR$. Then $x=ar$ and $y=as$ for some
$r,s\in R$. Then $x+y=ar+as=a(r+s)$, so $x+y\in aR$. Next, let $x=ar\in aR$
and let $z\in R$. Then $xz=arz=a(rz)\in aR$. Also, $zx=xz$ since $R$ is
commutative, so $zx\in aR$. Therefore, $aR$ is an ideal of $R$. This
construction generalizes the previous two examples. The ideal $aR$ is
typically called the ideal
\index{ideal generated by an element}%
\emph{generated by} $a$. It is often written as $(a)$.
\end{example}

\begin{example}
Let $R$ be any commutative ring, and let $a,b\in R$. Set
\[
I=\left\{  ar+bs:r,s\in R\right\}  .
\]
To see that $I$ is an ideal of $R$, first let $x,y\in I$. Then $x=ar+bs$ and
$y=ar^{\prime}+bs^{\prime}$ for some $r,s,r^{\prime},s^{\prime}\in R$. Then
\begin{align*}
x+y  &  =(ar+bs)+(ar^{\prime}+bs^{\prime})\\
&  =(ar+ar^{\prime})+(bs+bs^{\prime})\\
&  =a(r+r^{\prime})+b(s+s^{\prime})\in I
\end{align*}
by the associative and distributive properties. Next, let $x\in I$ and $z\in
R$. Again, $x=ar+bs$ for some $r,s\in R$. Then
\begin{align*}
xz  &  =(ar+bs)z=(ar)z+(bs)z\\
&  =a(rz)+b(sz).
\end{align*}
This calculation shows that $xz\in I$. Again, since $R$ is commutative,
$zx=xz$, so $zx\in I$. Thus, $I$ is an ideal of $R$. We can generalize this
example to any finite number of elements of $R$: given $a_{1},\ldots,a_{n}\in
R$, if
\[
J=\left\{  a_{1}r_{1}+\cdots+a_{n}r_{n}:r_{i}\in R\text{ for each }i\right\}
,
\]
then a similar argument will show that $J$ is an ideal of $R$. The ideal $J$
is typically referred to as the ideal generated by the elements $a_{1}%
,\ldots,a_{n}$, and it is often denoted by $(a_{1},\ldots,a_{n})$.
\end{example}

The division algorithm has a nice application to the structure of ideals of
$\mathbb{Z}$ or of $F[x]$. We prove the result for polynomials, leaving the
analogous result for $\mathbb{Z}$ to the reader.

\begin{theorem}
Let $F$ be a field. Then any ideal of $F[x]$ can be generated by a single
polynomial. That is, if $I$ is an ideal of $F[x]$, then there is a polynomial
$f$ with $I=(f)=\left\{  fg:g\in F[x]\right\}  $.
\end{theorem}

\begin{proof}
Let $I$ be an ideal of $F[x]$. If $I=\left\{  0\right\}  $, then $I=(0)$. So,
suppose that $I$ is nonzero. Let $f\in I$ be a nonzero polynomial of least
degree. We claim that $I=(f)$. To prove this, let $g\in I$. By the division
algorithm, there are polynomials $q,r$ with $g=qf+r$ and $\deg(r)<\deg(f)$.
Since $f\in I$, the product $qf\in I$, and thus $g-qf\in I$ as $g\in I$. We
conclude that $r\in I$. However, the assumption on the degree of $f$ shows
that the condition $\deg(r)<\deg(f)$ forces $r=0$. Thus, $g=qf\in(f) $. This
proves $I\subseteq(f)$. Since every multiple of $f$ is in $I$, the reverse
inclusion $(f)\subseteq I$ is also true. Therefore, $I=(f)$.
\end{proof}


We can give an ideal theoretic description of greatest common divisors in
$\mathbb{Z}$ and in $F[x]$. Suppose that $f$ and $g$ are polynomials over a
field $F$. If $\gcd(f,g)=d$, then we have proved that $d=fh+gk$ for some
polynomials $h,k$. Therefore, $d$ is an element of the ideal $I=\left\{
fs+gt:s,t\in F[x]\right\}  $. However, since $d$ divides $f$ and $g$, it
follows that $d$ divides every element of $I$. Therefore, $I=(d)$ is simply
the set of multiples of $d$. Therefore, one can identify the greatest common
divisor of $f$ and $g$ by identifying a monic polynomial $d$ satisfying
$I=(d)$.

We now use ideals to define quotient rings. In order to define them, we first
need to specify what are their elements. These are cosets, which we now
define. We have seen cosets when we discussed decoding with the Hamming code.
These cosets arose from a subspace of a vector space. The idea here is
essentially the same; the only difference is that we start with an ideal of a
ring instead of a subspace of a vector space.

\begin{definition}
Let $R$ be a ring and let $I$ be an ideal of $R$. If $a\in R$, then the coset
$a+I$ is defined as $a+I=\left\{  a+x:x\in I\right\}  $.
\end{definition}

Recall the description of equivalence classes for the relation congruence
modulo $n$. For example, if $n=5$, then we have five equivalence classes, and
they are
\begin{align*}
\overline{0}  &  =\left\{  0,5,10,\ldots,-5,-10,\ldots\right\}  ,\\
\overline{1}  &  =\left\{  1,6,11,\ldots,-4,-9,-14,\ldots\right\}  ,\\
\overline{2}  &  =\left\{  2,7,12,\ldots,-3,-8,-13,\ldots\right\}  ,\\
\overline{3}  &  =\left\{  3,8,13,\ldots,-2,-7,-12,\ldots\right\}  ,\\
\overline{4}  &  =\left\{  4,9,14,\ldots,-1,-6,-11,\ldots\right\}  .
\end{align*}
By the first example above, the set $5\mathbb{Z}$ of multiples of $5$ forms an
ideal of $\mathbb{Z}$. These five equivalence classes can be described as
cosets, namely,
\begin{align*}
\overline{0}  &  =0+5\mathbb{Z},\\
\overline{1}  &  =1+5\mathbb{Z},\\
\overline{2}  &  =2+5\mathbb{Z},\\
\overline{3}  &  =3+5\mathbb{Z},\\
\overline{4}  &  =4+5\mathbb{Z}.
\end{align*}
In general, for any integer $a$, we have $a+5\mathbb{Z}=\overline{a}$. Thus,
cosets for the ideal $5\mathbb{Z}$ are the same as equivalence classes modulo
$5 $. In fact, more generally, if $n$ is any positive integer, then the
equivalence class $\overline{a}$ of an integer $a$ modulo $n$ is the coset
$a+n\mathbb{Z}$ of the ideal $n\mathbb{Z}$.

We have seen that an equivalence classes can have different names. Modulo $5$,
we have $\overline{1}=\overline{6}$ and $\overline{2}=\overline{-3}%
=\overline{22}$, for example. Similarly, cosets can be represented in
different ways. If $R=F[x]$ and $I=xR$, the ideal of multiples of the
polynomial $x$, then $0+I=x+I=x^{2}+I=4x^{17}+I$. Also, $1+I=(x+1)+I$. For
some terminology, we refer to $a$ as a
\index{coset representative}%
\emph{coset representative} of $a+I$. One important thing to remember is that
the coset representative is not unique, as the examples above demonstrate.

When we defined operations on $\mathbb{Z}_{n}$, we defined them with the
formulas $\overline{a}+\overline{b}=\overline{a+b}$ and $\overline{a}%
\cdot\overline{b}=\overline{ab}$. Since these equivalence classes are the same
thing as cosets for $n\mathbb{Z}$, this leads us to consider a generalization.
If we replace $\mathbb{Z}$ by any ring and $n\mathbb{Z}$ by any ideal, we can
mimic these formulas to define operations on cosets. First, we give a name to
the set of cosets.

\begin{definition}
If $I$ is an ideal of a ring $R$, let $R/I$ denote the set of cosets of $I$.
In other words, $R/I=\left\{  a+I:a\in R\right\}  $.
\end{definition}

We now define operations on $R/I$ in a manner like the operations on
$\mathbb{Z}_{n}$. We define
\begin{align*}
\left(  a+I\right)  +\left(  b+I\right)   &  =(a+b)+I,\\
\left(  a+I\right)  \cdot\left(  b+I\right)   &  =(ab)+I.
\end{align*}
In other words, to add or multiply two cosets, first add or multiply their
coset representatives, then take the corresponding coset. As with the
operations on $\mathbb{Z}_{n}$, we have to check that these formulas make
sense. In other words, the name we give to a coset should not affect the value
we get when adding or multiplying. We first need to know when two elements
represent the same coset. To help with the proof, we point out two simple
properties. If $I$ is an ideal, then $0\in I$. Furthermore, if $r\in I $, then
$-r\in I$. The proofs of these facts are left as exercises.

\begin{lemma}
Let $I$ be an ideal of a ring $R$. If $a,b\in R$, then $a+I=b+I$ if and only
if $a-b\in I$.
\end{lemma}

\begin{proof}
Let $a,b\in R$. First suppose that $a+I=b+I$. From $0\in I$ we get $a=a+0\in
a+I$, so $a\in b+I$. Therefore, there is an $x\in I$ with $a=b+x$. Thus,
$a-b=x\in I$. Conversely, suppose that $a-b\in I$. If we set $x=a-b$, an
element of $I$, then $a=b+x$. This shows $a\in b+I$. So, for any $y\in I$, we
have $a+y=b+(x+y)\in I$, as $I$ is closed under addition. Therefore,
$a+I\subseteq b+I$. The reverse inclusion is similar; by using $-x=b-a$, again
an element of $I$, we will get the inclusion $b+I\subseteq a+I$, and so
$a+I=b+I$.
\end{proof}


In fact, we can generalize the fact that equivalence classes modulo $n$ are
the same thing as cosets for $n\mathbb{Z}$. Given an ideal, we can define an
equivalence relation by mimicing congruence modulo $n$. To phrase this
relation in a new way, $a\equiv b\operatorname{mod}n$ if and only if $a-b$ is
a multiple of $n$, so $a\equiv b\operatorname{mod}n$ if and only if $a-b\in
n\mathbb{Z} $. Thus, given an ideal $I$ of a ring $R$, we may define a
relation by $x\equiv y\operatorname{mod}I$ if $x-y\in I$. One can prove in the
same manner as for congruence modulo $n$ that this is an equivalence relation,
and that, for any $a\in R$, the coset $a+I$ is the equivalence class of $a$.

\begin{lemma}
Let $I$ be an ideal of a ring $R$. Let $a,b,c,d\in R$.

\begin{enumerate}
\item If $a+I=c+I$ and $b+I=d+I$, then $a+b+I=c+d+I$.

\item If $a+I=c+I$ and $b+I=d+I$, then $ab+I=cd+I$.
\end{enumerate}
\end{lemma}

\begin{proof}
Suppose that $a,b,c,d\in R$ satisfy $a+I=c+I$ and $b+I=d+I$. To prove the
first statement, by the lemma we have elements $x,y\in I$ with $a-c=x$ and
$b-d=y$. Then
\begin{align*}
(a+b)-(c+d)  & =a+b-c-d\\
& =(a-c)+(b-d)\\
& =x+y\in I.
\end{align*}
Therefore, again by the lemma, $(a+b)+I=(c+d)+I$. For the second statement, we
rewrite the equations above as $a=c+x$ and $b=d+y$. Then
\begin{align*}
ab  & =(c+x)(d+y)=c(d+y)+x(d+y)\\
& =cd+(cy+xd+xy).
\end{align*}
Since $x,y\in I$, the three elements $cy$, $xd$, $xy$ are all elements of $I$.
Thus, the sum $cy+xd+xy\in I$. This shows us that $ab-cd\in I$, so the lemma
yields $ab+I=cd+I$.
\end{proof}


The consequence of the lemma is exactly that our coset operations make sense.
Thus, we can ask whether or not $R/I$ is a ring. The answer is yes, and the
proof is easy, and is exactly parallel to the proof for $\mathbb{Z}_{n}$.

\begin{theorem}
Let $I$ be an ideal of a ring $R$. Then $R/I$, together with the operations of
coset addition and multiplication, forms a ring.
\end{theorem}

\begin{proof}
We have several properties to verify. Most follow immediately from the
definition of the operations and from the ring properties of $R$. For example,
to prove that coset addition is commutative, we see that for any $a,b\in R$,
we have
\begin{align*}
\left(  a+I\right)  +\left(  b+I\right)   & =(a+b)+I\\
& =(b+a)+I\\
& =\left(  b+I\right)  +\left(  a+I\right)  .
\end{align*}
This used exactly the definition of coset addition and commutativity of
addition in $R$. Most of the other ring properties hold for similar reasons,
so we only verify those that are a little different. For existence of an
additive identity, we have the additive identity $0$ of $R$, and it is natural
to guess that $0+I$ is the identity for $R/I$. To see that this is indeed
true, let $a+I\in R/I$. Then
\[
\left(  a+I\right)  +\left(  0+I\right)  =(a+0)+I=a+I.
\]
Thus, $0+I$ is the additive identity for $R/I$. Similarly $1+I$ is the
multiplicative identity, since
\[
\left(  a+I\right)  \cdot\left(  1+I\right)  =(a\cdot1)+I=a+I
\]
and
\[
\left(  1+I\right)  \cdot\left(  a+I\right)  =(1\cdot a)+I=a+I
\]
for all $a+I\in R/I$. Finally, the additive inverse of $a+I$ is $-a+I$ since
\[
\left(  a+I\right)  +\left(  -a+I\right)  =(a+(-a))+I=0+I.
\]
Therefore, $R/I$ is a ring.
\end{proof}


The ring $R/I$ is called a
\index{quotient ring}%
\emph{quotient ring} of $R$. This idea allows us to construct new rings from
old rings. For example, the ring $\mathbb{Z}_{n}$ is really the same thing as
the quotient ring $\mathbb{Z}/n\mathbb{Z}$, since we have identified the
equivalence classes modulo $n$; that is, the elements of $\mathbb{Z}_{n}$,
with the cosets of $n\mathbb{Z}$; i.e., the elements of $\mathbb{Z}%
/n\mathbb{Z}$. It is this construction applied to polynomial rings that we
will use to build extension fields. We recall Proposition \ref{Zpafield} above
that says $\mathbb{Z}_{n}$ is a field if and only if $n$ is a prime. To
generalize this result to polynomials, we first need to define the polynomial
analogue of a prime number.

\begin{definition}%
\index{irreducible polynomial}%
Let $F$ be a field. A nonconstant polynomial $f\in F[x]$ is said to be
irreducible over $F$ if whenever $f$ can be factored as $f=gh$, then either
$g$ or $h$ is a constant polynomial.
\end{definition}

