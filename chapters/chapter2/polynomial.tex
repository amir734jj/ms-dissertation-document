% \section{Polynomial ring \texorpdfstring{$K[x]$}{K[x]}}
\section{Quotient ring  }
In mathematics, especially in the field of abstract algebra, a polynomial ring or polynomial algebra is a ring formed from the set of polynomials in one or more indeterminate variable (traditionally also called variables) with coefficients in another ring, often a field.

% In this section we describe a method for producing field extension of a given field. If $F$ is a field, then a \textit{field extension} is a field $K$ that contains $F$. For example, $\mathbb{C}$ is a field extension of $\mathbb{R}$ since $\mathbb{C}$ is a field containing in $\mathbb{R}$. Similarly, $\mathbb{C}$ is a field extension of $\mathbb{Q}$. To produce a field extension of a field $F$ we will use a polynomial $p(x)$ with coefficients in $F$, and we will produce it by mimicking the idea of producing the integers modulo $n$ by starting with the integers and a fixed integer $n$. In order to do this we need to know that the arithmetic of polynomials is sufficiently similar to the arithmetic of integers. Let $K$ be a finite field extension of the field of rational numbers $\mathbb{Q}$. This means the degree of the field extension $[K : \mathbb{Q}]$ is finite. In other words, $K$ can be seen as a $\mathbb{Q}$-vector space of dimension $[K : \mathbb{Q}]$.

% \begin{definition}
% \normalfont
% An algebraic number field $K$ is a finite field extension of the field of rational numbers $\mathbb{Q}$.
% \end{definition}

% Let $K$ be an algebraic number field and for an element $x \in K$, $x$ is a root of a non-zero polynomial $f$ with coefficients in $\mathbb{Q}$ or $\mathbb{Z}$. If all coefficients of the polynomial are integers and the leading coefficient equals $1$, the element $x$ is called integral element of the algebraic number field $K$.


\begin{definition}
\normalfont
The polynomial ring, $K[x]$, in $x$ over a field $K$ is defined as the set of expressions, called polynomials in $x$, of the form $p(x)=p_{0}+p_{1}x+p_{2}x^{2}+\cdots +p_{m-1}x^{m-1}+p_{m}x^{m}$,
where $p_{0}, p_{1}, \dots, p_{m}$, the coefficients of $p$, are elements of $K$. The symbol $x$ is called an indeterminate or variable and degree of a polynomial $p$, written as deg($p$), is the largest $k$ such that the coefficient of $x^{k}$ is not zero.
\end{definition}




The polynomial ring in $x$ over $K$ is equipped with an addition, a multiplication and a scalar multiplication that make it a commutative algebra. These operations are defined according to the ordinary rules for manipulating algebraic expressions. Specifically, if $p=p_{0}+p_{1}x+p_{2}x^{2}+\cdots +p_{m}x^{m}$, and $q=q_{0}+q_{1}x+q_{2}x^{2}+\cdots +q_{n}x^{n}$, then $p+q=r_{0}+r_{1}x+r_{2}x^{2}+\cdots +r_{k}x^{k}$, and $pq=s_{0}+s_{1}x+s_{2}x^{2}+\cdots +s_{l}x^{l}$, where $k = \max(m, n)$, $l = m + n$, $r_{i}=p_{i}+q_{i}$ and $s_{i}=p_{0}q_{i}+p_{1}q_{i-1}+\cdots +p_{i}q_{0}$.







% The ring $K[x]$ of polynomials over $K$ is obtained from $K$ by adjoining one element, $x$. It turns out that any commutative ring $L$ containing $K$ and generated as a ring by a single element in addition to $K$ can be described using $K[x]$. In particular, this applies to finite field extensions of $K$.


\begin{definition}
\normalfont
Let $R$ be a ring. An ideal $I$ is a nonempty subset of $R$ such that (i) if
$a,b\in I$, then $a+b\in I$, and (ii) if $a\in I$ and $r\in R$, then $ar\in I$
and $ra\in I$.
\end{definition}
This definition says that an ideal is a subset of $R$ closed under addition that satisfies a strengthened form of closure under multiplication. Not only is the product of two elements of $I$ also in $I$, but the product of an element of $I$ and any element of $r$ is an element of $I$. Therefore we can define addition in the set $R/I = \{ a + I | a \in R\}$ by $(a + I)+(b + I)=(a + b) + I$ and multiplication by $(a + I)(b + I) = ab + I$. Hence, $R/I$ is called the \textit{quotient ring} or factor ring of $R$ by $I$. Let $L=K[x]/ ( p(x) )$ be the quotient ring of the polynomial ring $K[x]$ by the ideal generated by $p$ (i.e. $p$ is a univariate polynomial over a field $K$), then $L$ is a field if and only if $p$ is irreducible polynomial over $K$.

% just as $\mathbb{Z}^n$ and $F[x]/(p(x))$ were rings, so is 


% \subsection{Cyclotomic Number Fields}
% One example of algebraic number fields are cyclotomic number fields. They are the foundation
% of ideal-lattice-based cryptography. First, we define and study cyclotomic polynomials.

% \begin{definition}
% \normalfont
% The $m$-th cyclotomic polynomial $\Phi_m(x) \in \mathbb{Z}[x]$ is the polynomial

% \[ \Phi_{m}(x) = \prod_{k \in \mathbb{Z}_{m}^*} (x - \zeta^k_m)\]

% where $\zeta_m = e^{2\pi i / m}$
% \end{definition}

% An $m$-th root of unity is called primitive if it is not a $k$-th root of unity for some $k < m$. All $m$-th primitive roots are given by $\zeta_m^k$ for $k \in \mathbb{Z}_m^x$, where $\mathbb{Z}_m$ denotes $\mathbb{Z} / m \mathbb{Z}$. Hence, the $m$-th cyclotomic polynomial is the polynomial whose roots are the primitive $m$-th roots of unity in $\mathbb{C}$.



% $\Phi_m(x)$ is an irreducible polynomial with leading coefficient $1$. Hence, it is the minimal polynomial of $\zeta_m$.


% Euler's phi-function is denoted by $\phi(x)$ and counts the elements in $\mathbb{Z}_x$ that are relatively prime to $x$. The degree of $\Phi_m(x)$ is given by $\phi(m) = n$, since elements in $\mathbb{Z}_m$ have a multiplicative inverse if and only if they are relatively prime to $m$.



% Let $p \in \mathbb{Z}$ be a prime number. The cyclotomic polynomial
% $\Phi_p(x)$ can be written as \[ \Phi_p(x) = 1 + x + \dots + x^{p-1} \]

% \begin{lemma}
% \normalfont
% Let $m = p^k$. Then $\Phi_m(x) = \Phi_p(x^{m/p})$. In particular, $\Phi_{2^k}(x) = x^n + 1$, where $n = \phi(2^k) = 2^{k-1}$.
% \end{lemma}



% To measure the suitability of different polynomials $p$ to be used in Ring variant of LWE, Regev in \cite{cryptoeprint:2012:230} defined the \textit{expansion factor}. This gives an indication of how much the coefficients of a polynomial are expanded by reducing modulo $p(x)$. By examining this expansion factor, Regev has found several appropriate choices for $p(x)$, two of these are worth noting here: $x^{n-1} + x^{n-2} + \dots + x + 1$, where $n$ is prime; and $x^n + 1$, where $n$ is a power of two.

% Note that the first is one of the factors of $x^n + 1$ and that it is irreducible for prime $n$. Likewise, the second is irreducible when $n$ is a power of two. Of these two, $x^n + 1$ has half the expansion factor of $x^{n-1} +x^{n-2} + \dots +x+ 1$.

% In the following (ideal lattice subsection), a consequences of the choice $p = x^{n} + 1$ will be briefly examined.

