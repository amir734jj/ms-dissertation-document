\section{\texorpdfstring{$\mathrm{BCNS}$}{BCNS} and \texorpdfstring{$\mathrm{NewHope}$}{NewHope} diagram comparison}
The basic sketch of both $\mathrm{BCNS}$ and $\mathrm{NewHope}$ protocols are very similar to Ding's key-exchange protocol ($\mathrm{DXL}$). They all share the basics, initiator sends a polynomial $b$ to responder and responder replies by sending $b', c$ where $c$ is a reconciliation information or a bit string to improve success probability of key-agreement. For every coefficient of responder's calculated key, there is one reconciliation bit and there are at most $1024$ coefficients. Therefore, responder is sending at most $1024$ bits as a reconciliation information.

\begin{figure}[H]
    \centering
	\begin{tabular}{|lll|}
		\hline
		$a \leftarrow \text{Uniformly random}$          &                      &                                                                    \\\hline
		\textbf{Alice}                                  &                      & \textbf{Bob}                                                       \\\hline
		$s, e \xleftarrow{\$} \chi$                     &                      & $s', e' \xleftarrow{\$} \chi$                                      \\
		$b \leftarrow as + e$                           & $\xrightarrow{b}$    & $b' \leftarrow as' + e'$                                           \\
		                                                &                      & $e'' \xleftarrow{\$}\chi$                                          \\
		                                                &                      & $v \leftarrow bs' + e''$                                           \\
		                                                & $\xleftarrow{b', c}$ & $\bar{v} \xleftarrow{\$}\text{dbl(v)}$                            \\
		                                                &                      & $c \leftarrow \langle \bar{v} \rangle_{2q, 2} \in \{0, 1\}^{n}$    \\
% 		$k_{A} \leftarrow rec(2b's, c)\in \{0, 1\}^{n}$ &                      & $k_{B} \leftarrow \lfloor \bar{v} \rfloor_{2q, 2}\in \{0, 1\}^{n}$ \\\hline
	    $k_{A} \leftarrow \text{rec}(2b's, c)\in \{0, 1\}^{n}$ &                      & $k_B \leftarrow \text{rec}(2bs', c) \in \{0, 1\}^n$                       \\\hline
	\end{tabular}
	\caption{Diagram of $\mathrm{BCNS}$ protocol}
\end{figure}



In $\mathrm{NewHope}$ protocol however, initiator also sends a \textit{seed} derived from a uniformly random distribution that enables the other party to deterministically create a shared polynomial by themselves without actually sending it. This is results in an efficiency in bandwidth needed for key-exchange. Also, to make sure the resulting shared key is uniformly random or number of $0$'s and $1$'s are uniformly distributed, NewHope uses $\mathrm{SHA}$-3. Note that if number of $0$'s and $1$'s are always exactly equal then we can deterministically find the shared key. Using a hash function will not have any impact on hardness of Ring-$\mathrm{LWE}$, it is just matter of making bits of resulting shared key uniformly distributed and making a sure it is irreversible (i.e. eavesdropper needs to break $\mathrm{SHA}$-3 first and then would be able to break Ring-$\mathrm{LWE}$). 

Another important observation is reconciliation information of $\mathrm{NewHope}$ is slightly different from $\mathrm{BCNS}$ protocol. In NewHope, one party sends a sub-cell number calculated from coordinate tuple generated by every $4$ coefficients to other party. Further, $\mathrm{NewHope}$ protocol divides each Voronoi cell to $16$ sub-cells so for every $4$ coefficient it requires up-to $4$ bits. Considering that there are total of up-to $1024$ coefficients in responder's key then responder is in fact sending $256 \times 4 = 1024$ bits as a reconciliation information, which is the same number of bits as in $\mathrm{BCNS}$ protocol.

\begin{figure}[H]
	\centering
	\begin{tabular}{|lll|}
		\hline
		$q = 12289 < 2^{14}, n = 1024, \Psi_{16}$           &                          &                                                     \\\hline
		\textbf{Alice}                                      &                          & \textbf{Bob}                                        \\\hline
		$seed \xleftarrow{\$} \{0, 1\}^{256}$               &                          &                                                     \\
		$a \leftarrow \text{parse}(\text{SHAKE-128}(seed))$ &                          &                                                     \\
		$s, e \xleftarrow{\$} \Psi_{16}^{n}$                &                          & $s', e', e'' \xleftarrow{\$} \Psi_{16}^{n}$          \\
		$b \leftarrow as + e$                               & $\xrightarrow{(b, seed)}$& $a \leftarrow \text{parse}(\text{SHAKE-128}(seed))$ \\
		                                                    &                          & $u \leftarrow as' + e'$                             \\
		                                                    &                          & $v \leftarrow bs' + e''$                            \\
		$v' \leftarrow us$                                  & $\xleftarrow{(u, r)}$    & $r \xleftarrow{\$}\text{HelpRec}(v)$                \\
		$v \leftarrow \text{Rec}(v', r)$                    &                          & $v \leftarrow \text{Rec}(v, r)$                     \\
		$\mu \leftarrow \text{SHA3-256}(v)$                 &                          & $\mu \leftarrow \text{SHA3-256}(v)$                 \\\hline
	\end{tabular}
	\caption{Diagram of $\mathrm{NewHope}$ protocol}
\end{figure}


With regards to $e''$ added to Bob's calculated key and the reason for it, notice if $e''$ is not there, then $v=bs'$ in both protocols, which means it would be easy to recover $s'$ given $v$ and $b$. Since $b$ is sent in the clear over the channel, and a (randomized) function of $v$ appears in the clear as $c$ (or $r$), without using $e''$ to hide $s'$, information about $s'$ would likely be leaked both to Alice and to any eavesdropper of the channel. Even if it would not immediately leak all of $s'$, any leakage is clearly bad. As a result, $e''$ is used to make sure that $v = bs' + e''$ is indistinguishable from random, i.e. the distribution of $v$ is independent of $s'$, assuming Ring-$\mathrm{LWE}$ is hard. See also the security proof in the $\mathrm{BCNS}$ paper \cite{bos2015post}, where Game 1 and Game 2 are assumed to be indistinguishable under the Ring-$\mathrm{LWE}$ assumption. The reason Alice does not add some $e''$ as well is because Alice uses $e$ to hide her own secret $s$, just like Bob uses $e'$ to hide his secret $s'$. To agree on the key (both parties have different noise on the shared key, which they do not wish to disclose). Bob has to send this extra key reconciliation message, for which he again hides his secret $s'$ with fresh random noise $e''$.

